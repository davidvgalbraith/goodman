\documentclass[11pt, oneside]{article}   	% use "amsart" instead of "article" for AMSLaTeX format

\usepackage{geometry}                		% See geometry.pdf to learn the layout options. There are lots.
\usepackage{amsthm}
\usepackage{ textcomp }
\usepackage{ amssymb }
\usepackage{amsmath}
\usepackage{ wasysym }
\newcommand{\ints}{\mathbb{Z}}
\newcommand{\nats}{\mathbb{N}}
\newcommand{\reals}{\mathbb{R}}
\newcommand{\comps}{\mathbb{C}}
\newcommand{\rats}{\mathbb{Q}}
\newcommand{\gt}{G_{\text{tor}}}
\newcommand{\ann}{\text{ann}}
\newcommand{\inv}{^{-1}}
\newcommand{\be}{\begin{enumerate}}
\newcommand{\ee}{\end{enumerate}}
\newcommand{\fs}{f_1 \ldots f_k}
\newcommand{\poly}{\sum_na_nx^n}
\newcommand{\andd}{\text{ and }}
\newcommand{\ra}{\rightarrow}
\newcommand{\lra}{\leftrightarrow}
\newcommand{\ct}{\cos\theta}
\newcommand{\st}{\sin\theta}
\newcommand{\cycle}{(a_1, a_2, \ldots a_n)}
\newcommand{\picycle}{ (\pi(a_1), \pi(a_2) \ldots \pi(a_n))}
\newcommand{\tai}{\tilde{A_i}}
\newcommand{\dotp}[2]{\langle #1, #2 \rangle}
\newcommand{\threemat}[9]{\left( \begin{array}{ccc} #1 & #2 & #3 \\ #4 & #5 & #6 \\ #7 & #8 & #9 \end{array} \right)}
\newcommand{\twomat}[4]{\left( \begin{array}{ccc} #1 & #2 \\ #3 & #4 \end{array} \right)}
\newcommand{\twoid}{\twomat{1}{0}{0}{1}}
\newcommand{\threeid}{\threemat{1}{0}{0}{0}{1}{0}{0}{0}{1}}
\newcommand{\tp}{^{\text{T}}}
\newcommand{\ply}{R[x_1, \ldots x_n]}
\newcommand{\multisum}[1]{\sum_I #1_Ix^I}
\newcommand{\matmultdef}{(AB)_{ij} = \sum_kA_{ik}B_{kj}}
\newcommand{\tf}{\tilde{f}}
\newcommand{\mat}{\text{Mat}}
\newcommand{\cubic}[1]{#1^3 + a#1^2 + b#1 + c}
\newcommand{\kan}{K[a_1, a_2, \ldots a_n]}
\newcommand{\kanp}{K(a_1, a_2, \ldots a_n)}
\newcommand{\fix}{\text{Fix}}
\newcommand{\aut}{\text{Aut}}
\newcommand{\angles}[1]{\langle #1 \rangle}


\geometry{letterpaper}                   		% ... or a4paper or a5paper or ... 
%\geometry{landscape}                		% Activate for for rotated page geometry
%\usepackage[parfill]{parskip}    		% Activate to begin paragraphs with an empty line rather than an indent
\usepackage{graphicx}				% Use pdf, png, jpg, or eps§ with pdflatex; use eps in DVI mode
								% TeX will automatically convert eps --> pdf in pdflatex		
\usepackage{amssymb}

\title{Chapter 4}
\author{Dave}
%\date{}							% Activate to display a given date or no date

\begin{document}
\maketitle

\section{Rotations of Regular Polyhedra}
\be
\item \be
\item If you have a set of $m$ orthonormal vectors, where $m < n$, then you can get an $m+1$th by calculating the span of your current set of vectors, finding any vector not in that span, and then using the Gram Shmit Orthogonalization Process to make it orthonormal. The Gram Shmit orthogonalization process works by expressing your vector outside the span of your current set as a linear combination of the vectors in your current set and also an unknown new vector orthonormal to the set, and solving for this new vector. For example, if you have a vector $v_1$, then you find a vector $x$ that's not a scalar multiple of $v_1$, and you can safely say $x = a v_1 + bv_2$ where $v_2$ is an unknown vector orthonormal to $v_1$. You have $a = \dotp{x}{v_1}$ and $b = \dotp{x}{v_2}$ where we got those by taking the dot product with $v_1$ and $v2$ on both sides of our equation for $x$ and the orthonormality of our $v$. So $v_2 = \frac{x - \dotp{x}{v_1} v_1}{\dotp{x}{v_2}}$. And so you can't actually find that denominator without knowing $v_2$, but it's ok because you don't even need it since you know that $v_2$ is normal, so all you need is to get $x - \dotp{x}{v_1} v_1$ and divide by $|x - \dotp{x}{v_1} v_1|$ and that is your $v_2$. Similarly, to find a $v_3$ once we have a $v_1$ and a $v_2$, we just find a $y$ outside the span of $v_1$ and $v_2$ and say $y = \dotp{y}{v_1}v_1 + \dotp{y}{v_2}v_2 + \dotp{y}{v_3}v_3$ and solve for $v_3$.
\item $\frac{1}{\sqrt{3}}[1, 1, 1], \frac{1}{\sqrt{2}}[1, -1, 0], \frac{1}{\sqrt{6}}[-1, -1, 2]$ is an orthonormal set.
\item $\threemat{0}{0}{1}{1}{0}{0}{0}{1}{0}$ is the matrix, and it looks like it does because rotating $2\pi/3$ around $[1, 1, 1]$ sends the x-axis to the z-axis, the z-axis to the y-axis, and the y-axis to the x-axis, so it has the same effect on the x-, y-, and z-coordinates of any point (rotating them).
\ee
\item Each of the vectors for the vertices of our tetrahedron has every entry $\pm 1$ and one sign in common with and two signs distinct from each other vertex, so if you average any two of them you get an elementary basis vector. Computation reveals that matrix $A_j$ for the rotation by $\pi$ about the vector $e_j$ has $A_{jj} = 1$, $A_{ii} = -1$ if $i \not = j$, and $A_{ij} = 0$ if $i \not = j$; this makes sense because the symmetry is a reflection about the two basis vectors other than $e_j$. Multiplying a diagonal matrix by such a matrix flips the sign of two of the diagonal entries, so any product $A_iA_j$ either has two negative diagonal entries (if $i \not = j$) or equals the identity (if $i=j$). So these matrices generate a group of order four. 
\item Exercise 2 gave us the four possible matrices with entries $\pm 1$ and one or three 1s. That gives the $\threemat{\pm1}{0}{0}{0}{\pm 1}{0}{0}{0}{\pm 1}$ elements. Multiplying each of those by $R$ gives the $\threemat{0}{0}{\pm1}{\pm1}{0}{0}{0}{\pm1}{0}$ elements, and multiplying those by $R$ gives the $\threemat{0}{\pm1}{0}{0}{0}{\pm1}{\pm1}{0}{0}$ ones. Since the composition of symmetries is a symmetry, this gives us 12 distinct symmetries of the tetrahedron, which we know there are 12 symmetries so this is all of them.
\item The set of diagonal permutations $V$ is a normal subgroup of the transformations of the tetrahedron, because the product of diagonal matrices is diagonal and if $R^k$ is a rotation matrix and $X \in V$ then in the product $R^kVR^{-k}$, $(R^kV)$ is a cyclic permutation on the rows of $V$ and multiplying that by $R^{-k}$ gives the inverse permutation on the columns, but if you exchange rows $i$ and $j$ of a diagonal matrix and then exchange columns $j$ and $i$, the result is still diagonal. The matrices from Exercise 4.1.3 are the product of a matrix in $V$ and a power of $R$ (that's how we built them in exercise 3), and the only power of $R$ that is in $V$ is the identity, so the conditions of Coronary 3.2.5 are satisfied, so (the tetrahedral rotations) = ($V \rtimes$ the powers of $R$).
\item \be
\item A symmetry of the cube that is not a symmetry of the tetrahedron interchanges the two inscribed tetrahedrons, and the product of a symmetry that interchanges with a symmetry that doesn't interchange is another symmetry that interchanges, so if $s\not\in T$ then $\{st : t \in T\}$ gives us twelve different matrices that interchange; since we know that there are 24 total symmetries of the cube, these 12 interchangers plus the 12 non-interchangers that make up $T$ constitute the full symmetry group.
\item $X = \threemat{0}{1}{0}{-1}{0}{0}{0}{0}{1}$ implements a symmetry of the cube but not the tetrahedron, so $C = T \cup TX$ by part a, and the determinant of $X$ is 1 so the determinant of $QX$ for $Q \in T$ is also 1, and $QX$ has one nonzero entry in each row and column because $Q$ and $X$ do.
\item The set $V$ of diagonal permutations of determinant 1 are still a normal subgroup as they were in Exercise 4, and $C = V S_3$ since a signed permutation matrix is the product of an unsigned permutation matrix and a signed diagonal matrix, and the only unsigned permutation matrix that is diagonal is the identity, so the conditions of Coronary 3.2.5 are satisfied and $C \cong V \rtimes S_3$.
\ee
\item \be
\item The centroid of the face is the average of the vertices of the face, so if $\tau$ fixes vertices then it fixes the centroid. Any rotational symmetry is a rotation about some line, and unless it's the identity it has to pass through any points it fixes. So $\tau$ must be a symmetry about a line through the centroid of the face and the centroid of the figure, since all rotational symmetries fix the centroid.
\item The midpoint of the edge is the average of the vertices, so if $\tau$ fixes vertices of an edge then it fixes its midpoint. So by the same argument I used in a, $\tau$ is a rotation about a line through the midpoint of the edge in question and the centroid of the figure.
\ee
\item A transformation of $\reals^3$ is a symmetry of the tetrahedron only if it fixes the set of vertices of the tetrahedron so the rotational group of the tetrahedron is a subgroup of $S_4$. It has at least 12 elements, since we've found 12, and at most 24, since that is the cardinality of $S_4$. But there's at least one vertex-preserving transformation that isn't a rotational symmetry of the tetrahedron: let's say you label the vertices A, B, C, and D, and your transformation switches the positions of A and B but leaves C and D where they are. None of the symmetries we have found implements this, so the rotational group of the tetrahedron has at most 23 elements. But the order of a subgroup of $S_4$ must divide 24 by Coronary 2.5.9, so the rotational group of the tetrahedron has 12 elements, all of which we've found.
\item A rotational symmetry of the cube must preserve the inscribed tetrahedrons. That means it's either a rotational symmetry of the inscribed tetrahedrons or an interchange-plus-rotation on the inscribed tetrahedrons. We have accounted for all of these, so we've accounted for all the symmetries.
\ee 
\section{Rotations of the Dodecahedron and Icosahedron}
\be
\item The cube has no rotational symmetries of order 5, so the dodecahedron's rotational symmetries of order 5 can't fix the cubes. The dodecahedron's rotational symmetries of order 3 pass through a vertex and its opposite vertex, and the cube's rotational symmetries of order 3 are about its diagonals. Each vertex A of the dodecahedron is a vertex of two inscribed cubes, and the vertex across from A is also a vertex of the same cubes and the other side of a diagonal of those cubes. So the rotational symmetries of order 3 are also symmetries on these two cubes, but not on the other 3 cubes, which they must permute. Finally, the rotational symmetries of order 2 pass through the midpoint of an edge, as do those of the cube. Each edge is shared between two inscribed cubes, so those cubes are fixed by the rotation of order 2, while the other cubes are not. So that's all the symmetries.
\item Intensive computation reveals that the distance from a point in $A$ to the nearest other point in $A$, the distance from any point in $A$ to $[0, 0, \sqrt{5}/2]$, the distance from a point in $A$ to the nearest point in $B$, the distance from a point in $B$ to the nearest other point in $B$, and the distance from any point in $B$ to $[0, 0, -\sqrt{5}/2]$ are all $\frac{\sqrt{10-2\sqrt{5}}}{2}$. So this figure is a regular polyhedron with 12 vertices and triangular faces, also known as an icosahedron. 
\item Yeah.
\ee
\section{What about Reflections?}
\be
\item Let $a$ and $b$ be the unit vectors defining the plane and $c$ be a unit vector perpendicular to them. Then for any vector $x \in \reals^3$, we have $x = \dotp{x}{a}a + \dotp{x}{b}b + \dotp{x}{c}c$, so $x - 2\dotp{x}{c}c = \dotp{x}{a}a + \dotp{x}{b}b - \dotp{x}{c}c$ which is as far away from the plane as $x$ but on the other side, so it's the reflection.
\item Let $\alpha = (x, y, z)$, then $J_\alpha = \threemat{1-2x^2}{-2xy}{-2xz}{-2xy}{1-2y^2}{-2yz}{-2xz}{-2yz}{1-2z^2}$.
\item To reflect $x$ across this plane, subtract $x_0$ from it, reflect the result across the same plane (since the normal vector is unchanged by translation), and then add $x_0$ back. The resulting formula is $T(x) = x - 2\dotp{x - x_0}{\alpha}$.
\item \be
\item $JA = J_b, JB = J_a, JR^2 = -E, JR = JR, JR^3 = JR^3, JE = J, JC = J_c, JD = J_d$, and $JJ = E$.
\item Yep, sure does.
\item We can use the commutativity to turn any product of rotation matrices and $J$s into a power of $J$ and a product of rotations, which since $J^2 = E$ and the rotations are a group equals either $J$ times a single rotation or a single rotation.
\item We verified in part c that the group is closed under multiplication, and the inverse of $JX$ where $X$ is a rotation is $JX\inv$ which is in our group since the rotations are a group.
\ee
\item \be
\item $J_aS = R, J_bS = R^3, J_rS = D, R^2S = J_c, AS = J_rR^3, ES = S, BS = J_rR, -ES = C$.
\item The product $XS$ is $X$ but with its first and second rows switched, whereas the product $SY$ is $Y$ but with the first and second columns switched, so $D'$ is the result of switching $D_{11}$ and $D_{22}$. So a product of matrices in our group is either a product of matrices in $\mathcal{D}$, in which case the product is easy to compute, or a product of a matrix $A \in \mathcal{D}$ and a matrix $BS \in \mathcal{D}S$, $ABS = (AB)S \in \mathcal{D}S$ or $BSA = (BA')S \in \mathcal{D}S$, or a product of a matrices $AS, BS \in \mathcal{D}S$, $ASBS = (AB')(SS) = AB' \in \mathcal{D}.$
\ee
\ee
\section{Linear Isometries}
\be
\item \be
\item We found its matrix in Exercise 4.3.2; extensive computation shows that its columns are orthonormal, so it's isometric. 
\item $\det \threemat{1-2x^2}{-2xy}{-2xz}{-2xy}{1-2y^2}{-2yz}{-2xz}{-2yz}{1-2z^2} = (1 - 2x^2)((1 - 2y^2)(1 - 2z^2) - 4y^2z^2) + 2xy(-2xy(1 - 2z^2) - 4xyz^2) - 2xz(-2xyz^2 + 2xz(1 - 2y^2)) = 1 - 2x^2 - 2y^2 - 2z^2 + 4x^2y^2 + 4x^2z^2 - 4x^2y^2 - 4x^2z^2  = 1 - 2(x^2 + y^2 + z^2) = 1 - 2 = -1$.
\item $j_{\tau(\alpha)}(y) = y - 2\dotp{y}{\tau(\alpha)} \tau(\alpha)$. Let $x = \tau\inv(y)$, then $\tau(j_\alpha(\tau\inv(y))) = \tau(x - 2\dotp{x}{\alpha}\alpha) = y - 2\dotp{x}{\alpha}\tau(\alpha) = y - 2\dotp{y}{\tau(\alpha)}\tau(\alpha)$ since $\tau$ is an isometry.
\item Orthogonal matrices implement linear isometries, so this is equivalent to c.
\item That's because change of basis is a similarity transformation.
\ee
\item Let $f$ and $g$ be isometries on $\reals^n$, then $d(f(g(x)), f(g(y))) = d(g(x), g(y)) = d(x, y)$, and $d(f\inv(x), f\inv(y)) = d(f(f\inv(x)), f(f\inv(y))) = d(x, y)$.  
\item $(AA\tp)_{ij} = \sum_kA_{ik}A\tp_{kj} = \sum_kA_{ik}A_{jk}$ which is the dot product of the $i$th and $j$th rows of $A$ so it is $\delta_{ij}$ if and only if the rows of $A$ are an orthonormal basis and similarly $(A\tp A)_{ij} = \delta_{ij}$ if and only if the columns of $A$ are an orthonormal basis.
\item \be
\item The matrix of reflection about that line is the matrix $J_\alpha$ where $\alpha$ is orthonormal to $[\cos\theta, \sin\theta]$ so let $\alpha = [-\sin\theta, \cos\theta]$ then $j\alpha([1, 0]) = [1 - 2\sin^2\theta, 2\sin\theta\cos\theta]$ and $j_\alpha([0, 1]) = [2\cos\theta\sin\theta, 1-2\cos^2\theta]$. Using trig identities, we get $J_\alpha = \twomat{\cos2\theta}{\sin2\theta}{\sin2\theta}{-\cos2\theta}$ = $R_{2\theta}J$.
\item By orthonormality, every matrix in O$(2, \reals)$ has the form $\twomat{\cos\theta}{\pm\sin\theta}{\sin\theta}{\pm\cos\theta}$ where the signs of the $\pm$ elements have to be different. We found that the ones where the cosine is the negative one are the reflection matrices, and computing the determinant of such matrices gets you -1.
\item $J_{\theta_1}J_{\theta_2} = \twomat{\cos2\theta_1}{\sin2\theta_1}{\sin2\theta_1}{-\cos2\theta_1} \twomat{\cos2\theta_2}{\sin2\theta_2}{\sin2\theta_2}{-\cos2\theta_2} = $\\ $ \twomat{\cos2\theta_1\cos2\theta_2 + \sin2\theta_1\sin2\theta_2}{\cos2\theta_1\sin2\theta_2 - \sin2\theta_1\cos2\theta_2}{\sin2\theta_1\cos2\theta_2 - \cos2\theta_1\sin2\theta_2}{\sin2\theta_1\sin2\theta_2 + \cos2\theta_1\cos2\theta_2} = $ \\ $\twomat{\cos2(\theta_2 - \theta_1)}{\sin2(\theta_2 - \theta_1)}{-\sin2(\theta_2 - \theta_1)}{\cos2(\theta_2 - \theta_1)}$
\item $R_\theta JR_0J = R_\theta J^2 = R_\theta$.
\ee
\item A matrix in SO($3, \reals$) is a matrix like $\threemat{1}{0}{0}{0}{\cos\theta}{-\sin\theta}{0}{\sin\theta}{\cos\theta}$, which is the product $\threemat{1}{0}{0}{0}{\cos\theta}{\sin\theta}{0}{\sin\theta}{-\cos\theta} \threemat{1}{0}{0}{0}{1}{0}{0}{0}{-1}$, which both of those factors are reflections.
\ee
\section{The Full Symmetry Group and Chirality}
\be
\item $i^2 = (-E)^2 = E^2 = E$, and $ir$ is $r$ with all its rows negated while $ri$ is $r$ with all its columns negated, which are the same thing. $\{e, i\} \cong \ints_2$, and $i$ is not a rotation so $\{e, i\} \cap R = e$. Both groups are normal, since $i$ commutes. So the conditions of Proposition 3.1.5 are satisfied, and $G \cong R \times \ints_2$.
\item $-e$ is a reflection symmetry of the dodecahedron, and it commutes with the rotations for the same reason it did in Exercise 1. So the subgroups are still normal, and $G \cong A_5 \times \ints_2$, where $G$ is the group of symmetries of the dodecahedron. This group is not isomorphic to $S_5$ because $S_5$ has, by my count, 25 elements whose order is 2 (10 2-cycles and 15 products of 2 disjoint 2-cycles) whereas $A_5$ only has those 15 products of 2 disjoint 2-cycles so $A_5 \times \ints_2$ has 30 elements of order 2 since for each of those 15 elements $s$ of order 2, both $(s, 0)$ and $(s, 1)$ have order 2.
\item A transformation of the tetrahedron is a symmetry only if it fixes the vertices, and each permutation of the vertices can be implemented by a rotation or a reflection-rotation. The possible arrangements of vertices correspond to $S_4$. 
\item The rotation subgroup consisting of matrices with determinant 1 is a normal subgroup since the composition of two reflection-rotations is a rotation, and another subgroup $\{e, -e\}$ is normal because its elements are in the center of the symmetry group, and their intersection is $e$ so $G \cong $(the symmetries of the card) $\times \ints_2$.
\item With the same argument used in Exercise 4, this group is $D_4 \times \ints_2$.
\item Similarly, this group is $D_n \times \ints_2$.
\ee
\end{document}  