\documentclass[11pt, oneside]{article}   	% use "amsart" instead of "article" for AMSLaTeX format

\usepackage{geometry}                		% See geometry.pdf to learn the layout options. There are lots.
\usepackage{amsthm}
\usepackage{ textcomp }
\usepackage{ amssymb }
\usepackage{amsmath}
\usepackage{ wasysym }
\newcommand{\ints}{\mathbb{Z}}
\newcommand{\nats}{\mathbb{N}}
\newcommand{\reals}{\mathbb{R}}
\newcommand{\comps}{\mathbb{C}}
\newcommand{\rats}{\mathbb{Q}}
\newcommand{\gt}{G_{\text{tor}}}
\newcommand{\ann}{\text{ann}}
\newcommand{\inv}{^{-1}}
\newcommand{\be}{\begin{enumerate}}
\newcommand{\ee}{\end{enumerate}}
\newcommand{\fs}{f_1 \ldots f_k}
\newcommand{\poly}{\sum_na_nx^n}
\newcommand{\andd}{\text{ and }}
\newcommand{\ra}{\rightarrow}
\newcommand{\lra}{\leftrightarrow}
\newcommand{\ct}{\cos\theta}
\newcommand{\st}{\sin\theta}
\newcommand{\cycle}{(a_1, a_2, \ldots a_n)}
\newcommand{\picycle}{ (\pi(a_1), \pi(a_2) \ldots \pi(a_n))}
\newcommand{\tai}{\tilde{A_i}}
\newcommand{\dotp}[2]{\langle #1, #2 \rangle}
\newcommand{\threemat}[9]{\left( \begin{array}{ccc} #1 & #2 & #3 \\ #4 & #5 & #6 \\ #7 & #8 & #9 \end{array} \right)}
\newcommand{\twomat}[4]{\left( \begin{array}{ccc} #1 & #2 \\ #3 & #4 \end{array} \right)}
\newcommand{\twoid}{\twomat{1}{0}{0}{1}}
\newcommand{\threeid}{\threemat{1}{0}{0}{0}{1}{0}{0}{0}{1}}
\newcommand{\tp}{^{\text{T}}}
\newcommand{\ply}{R[x_1, \ldots x_n]}
\newcommand{\multisum}[1]{\sum_I #1_Ix^I}
\newcommand{\matmultdef}{(AB)_{ij} = \sum_kA_{ik}B_{kj}}
\newcommand{\tf}{\tilde{f}}
\newcommand{\mat}{\text{Mat}}
\newcommand{\cubic}[1]{#1^3 + a#1^2 + b#1 + c}
\newcommand{\kan}{K[a_1, a_2, \ldots a_n]}
\newcommand{\kanp}{K(a_1, a_2, \ldots a_n)}
\newcommand{\fix}{\text{Fix}}
\newcommand{\aut}{\text{Aut}}
\newcommand{\angles}[1]{\langle #1 \rangle}


\geometry{letterpaper}                   		% ... or a4paper or a5paper or ... 
%\geometry{landscape}                		% Activate for for rotated page geometry
%\usepackage[parfill]{parskip}    		% Activate to begin paragraphs with an empty line rather than an indent
\usepackage{graphicx}				% Use pdf, png, jpg, or eps§ with pdflatex; use eps in DVI mode
								% TeX will automatically convert eps --> pdf in pdflatex		
\usepackage{amssymb}

\title{Chapter 4}
\author{Dave}
%\date{}							% Activate to display a given date or no date

\begin{document}
\maketitle

\section{Rotations of Regular Polyhedra}
\be
\item \be
\item If you have a set of $m$ orthonormal vectors, where $m < n$, then you can get an $m+1$th by calculating the span of your current set of vectors, finding any vector not in that span, and then using the Gram Shmit Orthogonalization Process to make it orthonormal. The Gram Shmit orthogonalization process works by expressing your vector outside the span of your current set as a linear combination of the vectors in your current set and also an unknown new vector orthonormal to the set, and solving for this new vector. For example, if you have a vector $v_1$, then you find a vector $x$ that's not a scalar multiple of $v_1$, and you can safely say $x = a v_1 + bv_2$ where $v_2$ is an unknown vector orthonormal to $v_1$. You have $a = \dotp{x}{v_1}$ and $b = \dotp{x}{v_2}$ where we got those by taking the dot product with $v_1$ and $v2$ on both sides of our equation for $x$ and the orthonormality of our $v$. So $v_2 = \frac{x - \dotp{x}{v_1} v_1}{\dotp{x}{v_2}}$. And so you can't actually find that denominator without knowing $v_2$, but it's ok because you don't even need it since you know that $v_2$ is normal, so all you need is to get $x - \dotp{x}{v_1} v_1$ and divide by $|x - \dotp{x}{v_1} v_1|$ and that is your $v_2$. Similarly, to find a $v_3$ once we have a $v_1$ and a $v_2$, we just find a $y$ outside the span of $v_1$ and $v_2$ and say $y = \dotp{y}{v_1}v_1 + \dotp{y}{v_2}v_2 + \dotp{y}{v_3}v_3$ and solve for $v_3$. 
\item
\item
\ee
\item 
\item 
\item 
\item 
\item 
\item 
\item
\ee 
\section{Rotations of the Dodecahedron and Icosahedron}
\be
\item 
\item 
\item 
\ee
\section{What about Reflections?}
\be
\item 
\item 
\item 
\item 
\item 
\ee
\section{Linear Isometries}
\be
\item 
\item 
\item 
\item 
\item 
\ee
\section{The Full Symmetry Group and Chirality}
\be
\item 
\item 
\item 
\item 
\item 
\item 
\ee
\end{document}  