\documentclass[11pt, oneside]{article}   	% use "amsart" instead of "article" for AMSLaTeX format
\usepackage{geometry}                		% See geometry.pdf to learn the layout options. There are lots.
\usepackage{amsthm}
\usepackage{ textcomp }
\usepackage{ amssymb }
\usepackage{amsmath}
\usepackage{ wasysym }
\newcommand{\ints}{\mathbb{Z}}
\newcommand{\nats}{\mathbb{N}}
\newcommand{\reals}{\mathbb{R}}
\newcommand{\comps}{\mathbb{C}}
\newcommand{\rats}{\mathbb{Q}}
\newcommand{\gt}{G_{\text{tor}}}
\newcommand{\ann}{\text{ann}}
\newcommand{\inv}{^{-1}}
\newcommand{\be}{\begin{enumerate}}
\newcommand{\ee}{\end{enumerate}}
\newcommand{\fs}{f_1 \ldots f_k}
\newcommand{\poly}{\sum_na_nx^n}
\newcommand{\andd}{\text{ and }}
\newcommand{\ra}{\rightarrow}
\newcommand{\lra}{\leftrightarrow}
\newcommand{\ct}{\cos\theta}
\newcommand{\st}{\sin\theta}
\newcommand{\cycle}{(a_1, a_2, \ldots a_n)}
\newcommand{\picycle}{ (\pi(a_1), \pi(a_2) \ldots \pi(a_n))}
\newcommand{\tai}{\tilde{A_i}}
\newcommand{\dotp}[2]{\langle #1, #2 \rangle}
\newcommand{\threemat}[9]{\left( \begin{array}{ccc} #1 & #2 & #3 \\ #4 & #5 & #6 \\ #7 & #8 & #9 \end{array} \right)}
\newcommand{\twomat}[4]{\left( \begin{array}{ccc} #1 & #2 \\ #3 & #4 \end{array} \right)}
\newcommand{\twoid}{\twomat{1}{0}{0}{1}}
\newcommand{\threeid}{\threemat{1}{0}{0}{0}{1}{0}{0}{0}{1}}
\newcommand{\tp}{^{\text{T}}}
\newcommand{\ply}{R[x_1, \ldots x_n]}
\newcommand{\multisum}[1]{\sum_I #1_Ix^I}
\newcommand{\matmultdef}{(AB)_{ij} = \sum_kA_{ik}B_{kj}}
\newcommand{\tf}{\tilde{f}}
\newcommand{\mat}{\text{Mat}}
\geometry{letterpaper}                   		% ... or a4paper or a5paper or ... 
%\geometry{landscape}                		% Activate for for rotated page geometry
%\usepackage[parfill]{parskip}    		% Activate to begin paragraphs with an empty line rather than an indent
\usepackage{graphicx}				% Use pdf, png, jpg, or eps§ with pdflatex; use eps in DVI mode
								% TeX will automatically convert eps --> pdf in pdflatex		
\usepackage{amssymb}

\title{Chapter 6}
\author{Dave}
%\date{}							% Activate to display a given date or no date

\begin{document}
\maketitle
\section{A Recollection of Rings}
\be
\item If $e$ and $e'$ are both identity elements in a ring $R$ then $e = e'e = e'$. And if $rr' = e = r''r$ for some $r \in R$, then multiplying the equality $rr' = e$ on the left by $r''$ gives $r''rr' = r' \ra r'' = r'$.
\item The additive inverse of $\sum_I a_Ix^I$ is $\sum_I -a_Ix^I \in \ply$ since $R$ is a ring. Similarly, the sum and product of elements of $\ply$ are in $\ply$. For distributivity of multiplication over addition, $\multisum{a}(\multisum{b} + \multisum{c}) = \multisum{a}(\sum_J(b_J + c_J)x^J) = \sum_I\sum_J a_I(b_J + c_J)x^{I + J} = \sum_I\sum_Ja_Ib_Jx^{I + J} + \sum_I\sum_J a_Ic_Jx^{I + J} = \multisum{a}\multisum{b} + \multisum{a}\multisum{c}$. Finally, for associativity of multiplication, $\multisum{a}(\multisum{b}\multisum{c}) = \multisum{a}(\sum_I\sum_Jb_Ic_Jx^{I + J}) = \sum_I\sum_J\sum_K a_Ib_Jc_Kx^{I + J + K} = (\sum_I\sum_Ja_Ib_Jx^{I + J})\multisum{c} = (\multisum{a}\multisum{b})\multisum{c}$. Finally, the multiplicative identity in $\ply$ is the same as that in $R$.
\item Let $R$ be the group of infinite matrices with finitely many nonzero elements. If $A, B \in R$, then $A + B$ has at most as many nonzero elements as the total such elements in $A$ and $B$, which is finite. The additive inverse of $A$ is $-A$, which has as many nonzero entries as $A$ since $-x = 0 \lra x=0$ for $x\in\reals$. $\matmultdef$, which if only finitely many $A_{ik}$ and $B_{kj}$ are nonzero then only finitely many sums of their products are nonzero. Since $R$ is a subset of the ring of real-valued infinite matrices, this suffices to prove it a subring.
\item If $A$ and $B$ are upper triangular, then $(A+B)_{ij} = A_{ij} + B_{ij} = 0 + 0 = 0$ if $i  > j$, so the sum of upper triangular matrices is upper triangular. And $\matmultdef$, so if $i < j$ then $A_{ik} = 0$ for $k = 1 \ldots j-1$ and $B_{kj} = 0$ for $k = j+1 \ldots n$ and each product $A_{ik}B_{kj}$ has at least one of these zeros so the product of upper triangular matrices is upper triangular. The additive inverse of an upper triangular matrix is upper triangular since -0 = 0, so the upper triangular matrices are a subring of the real matrices. The identity matrix is upper triangular.

For the upper triangular matrices with zeros on the diagonal, the arguments of the above paragraph go through unchanged except with $i < j$ changed to $i\le j$, and of course the identity does not have zeros on its main diagonal.
\item A linear combination of integers is an integer, so the set of integer-valued matrices is closed under sum and product. The same statement applies to natural numbers. The integer matrices are a subring while the natural matrices are not, because $-z \in \ints$ for all $z \in \ints$, whereas $-n \not \in \nats$ for any $n \in \nats$, so the integer matrices are closed under additive inverse while the naturals are not.
\item Let $R$ be the set of symmetric polynomials, and let $a, b \in R$. Then $a(x, y, z) = a(\pi(x, y, z)), b(x, y, z) = b(\pi(x, y, z))$ for all $\pi \in S_3$ $\lra$ $(a + b)(x, y, z) = a(x, y, z) + b(x, y, z) = a(\pi(x, y, z)) + b(\pi(x, y, z)) = (a + b)(\pi(x, y, z))$, so $R$ is closed under sums. A similar proof holds for products, and having established products we get additive inverses for free, since the additive inverse of $a$ is $-1 * a$, and $-1$ is a symmetric polynomial. This suffices to demonstrate that $R$ is a subring of the three-variabled polynomials.
\item I like chikin.
\item Compositions, sums, and inverses of linear maps are linear, so the linear maps on $V$ are a subring of the functions on $V$.
\item Let $e \in R$ be an identity, and let $r \in R$ with $f(r) = s$ where $f$ is our isomorphism and $s \in S$. Then $s = f(r) = f(er) = f(e)f(r) \ra f(e)$ is the identity in $S$. Since $f\inv$ is also a ring isomorphism, this proof is bidirectional. Similarly, $rr' = r'r \lra f(rr') = f(r'r) \lra f(r)f(r') = f(r')f(r)$, and since $f$ is a bijection, it is surjective so we can find $r, r' \in R$ such that $f(r) = s, f(r') = s'$ for any $s, s' \in S$.
\item Let $\{A_\alpha\}$ be a collection of subrings of $R$. Then $x, y \in \cap_\alpha A_\alpha \ra x, y \in A_\alpha \forall \alpha \ra x + y, xy, -x \in A_\alpha \forall \alpha$ since each $A_\alpha$ is a subring, so $xy, x+y, -x \in \cap_\alpha A_\alpha \ra \cap_\alpha A_\alpha$ is a subring. 

Let $Q$ be the smallest subring that contains $S$, then $R' \cap Q = Q$ for each $R'$ since otherwise the intersection would be a smaller subring containing $S$ since both $Q$ and $R'$ contain $S$, so since $Q$ is one of the $R'$, the intersection of all of them is equal to $Q$.
\item $p/q + p'/q' = (pq' + p'q) / qq' \in R(x), (p/q)(p'/q') = pp'/qq' \in R(x), -(p/q) = -p/q \in R(x)$, so $R(x)$ is a ring. The multiplicative inverse of $p/q$ is $q/p \in R(x)$, so $\reals(x)$ is a field.
\item Defining addition and multiplication pointwise, the closure of the set of functions from $X$ to $R$ under sum, product and additive inverse follows from those properties of $R$. $rr' = r'r$ for all $r, r' \in R \ra f(x)g(x) = g(x)f(x) \forall x \in X \forall f, g: X \to R$ since $f(x), g(x) \in R$. The reverse implication follows from the fact that $R$ is isomorphic to the constant-valued functions on $X$, which is trivial since we defined our operations on functions pointwise and the constant functions take only a single value. If $R$ has an identity, then the constant function on $X$ whose value is that identity is the identity in the function ring, and the units in the function ring are the functions whose range is comprised of units.
\item Let $T, T' \in S'$, then $STT' = TT'S, S(T + T') = ST + ST' = TS + T'S = (T + T')S, $ and $S(-T) = -ST = -TS = (-T)S$, so $S'$ is closed under sums, products and additive inverses, so it is a subring of End$_K(V)$.
\item Let $X$ be our set of linear combinations. Then $\sum n_{1g}g + \sum n_{2g} g = \sum (n_{1g} + n_{2g})g \in X, -\sum n_gg = \sum n_{-g}g \in X$, and $(\sum n_{1g}g)(\sum n_{2g}g) = \sum_g\sum_h n_{1g}n_{2h}gh \in X$, so $X$ is a subring of $GL(V)$.
\item $\ints G$ is an abelian additive group as long as $G$ is, since the inverse of $\sum_{g\in G}a_gg$ is $\sum_{g\in G} -a_gg \in \ints G$ and the other group properties are even more self-evident. Additionally, multiplication is associative, since $\sum a_gg(\sum b_hh * \sum c_i i) = \sum_g a_gg (\sum_h\sum_i b_hc_ihi) = \sum_g \sum_h \sum_i a_gb_hc_ighi = (\sum_g\sum_h a_gb_hgh) \sum_i c_ii =  (\sum a_gg * \sum b_hh) \sum c_i i$, and multiplication is distributive over addition, since $\sum a_gg(\sum b_hh + \sum c_ii) = \sum_ga_gg(\sum_h (b_h + c_h)h) = \sum_g\sum_h a_g(b_h+c_h)gh = \sum_g\sum_h a_gb_hgh + a_gc_hgh = \sum a_gg\sum b_hh + \sum a_gg \sum c_ii$.
\item Let $f$ be our isomorphism, then $f(a + bi + c + di) = (a + b + c + d, a + c - b - d) = (a + b, a - b) + (c + d, c - d) = f(a + bi) + f(c + di)$, and $f((a + bi)(c + di)) = f(ac + bd + adi + bci) = (ac + bd + ad + bc, ac + bd - ad - bc) = (a + b, a - b)(c + d, c - d) = f(a + bi)f(c + di)$ where $i = \xi$. So $f$ respects ring operations, and it's a bijection since its inverse is $f\inv(x, y) = ((x + y) / 2, (x - y) / 2)$, so it's a ring isomorphism.
\item The proofs are the same as those for $R[x]$, just with infinities as the upper limit of the sums.
\item 0(00) = 0 = (00)0, 0 + 0 = 0, 0(0 + 0) = 0 = 00 + 00, etc. so the zero ring is a ring. If 1=0, then for all $x \in S, x = x1 = x0 = 0$, so $S = \{0\}$.
\ee 
\section{Homomorphisms and Ideals}
\be
\item Let $f(A) = \twomat{A}{0}{0}{A}$, then $f(AB) = \twomat{AB}{0}{0}{AB} = \twomat{A}{0}{0}{A}\twomat{B}{0}{0}{B} = f(A)f(B)$, and $f(A + B) = \twomat{A+B}{0}{0}{A+B} =  \twomat{A}{0}{0}{A} + \twomat{B}{0}{0}{B} = f(A) + f(B).$ So $f$ is a homomorphism, and $f(I_2) = I_4$, so it's unital. Let $g(A) =  \twomat{A}{0}{0}{0}$, then similar computations give that $g$ is a homomorphism.
\item $\phi(a + bx + cx^2)(d + ex + fx^2) = \phi(ad + aex + afx^2 + bdx + bex^2 + cdx^2) = \threemat{ad}{ae + bd}{af + be + cd}{0}{ad}{ae + bd}{0}{0}{ad} = \threemat{a}{b}{c}{0}{a}{b}{0}{0}{a}\threemat{d}{e}{f}{0}{d}{e}{0}{0}{d}$ = $\phi(a + bx + cx^2)\phi(d+ex+fx^2)$, and $\phi(a + bx + cx^2 + d + ex + fx^2) = \threemat{a+d}{b+e}{c+f}{0}{a+d}{b+e}{0}{0}{a+d} = \threemat{a}{b}{c}{0}{a}{b}{0}{0}{a} + \threemat{d}{e}{f}{0}{d}{e}{0}{0}{d} = \phi(a + bx + cx^2) + \phi(d + ex + fx^2)$, so $\phi$ respects ring operations, so it is a homomorphism, and $\phi(1) = \threeid$, so $\phi$ is unital. Its colonel is the polynomials of degree $> 3$, since those have zeros for the coefficients $a_0, a_1$ and $a_2$.
\item $f(1)^2 = f(1^2) = f(1)$, and $f(x) = f(1x) = f(x1) = f(1x1) \ra ef(x) = f(x)e=ef(x)e$.
\item If $x=f(a), y=f(b) \in S,$ then $x + y = f(a) + f(b) = f(a + b) \in f(R), xy = f(a)f(b) = f(ab) \in f(R), $ and $-x = -f(a) = f(-a) \in f(R)$, so $f(R)$ is a subring of $S$.
\item $f(g(xy)) = f(g(x)g(y)) = f(g(x))f(g(y))$ and $f(g(x + y)) = f(g(x) + g(y)) = f(g(x)) + f(g(y))$, so $f \circ g$ is a ring homomorphism.
\item For $f \in I$, $f(x) = 0$ for $x \in S$, so for $g \in R$, $f(x)g(x) = 0g(x) = 0$, so $I$ is an ideal.
\item $\threemat{0}{a}{b}{0}{0}{c}{0}{0}{0}\threemat{d}{e}{f}{0}{g}{h}{0}{0}{i} = \threemat{0}{ag}{ah + bi}{0}{0}{ci}{0}{0}{0} \in I$, so $I$ is an ideal.
\item This follows from Proposition 6.2.25, setting $\mathcal{S} = \{x\}$.
\item This follows from Proposition 6.2.27, setting $\mathcal{S} = \{x\}$.
\item Let $I$ be a nonzero ideal in a field $F$, and $x \not=0 \in I$, then for all $y \in F$, $xx\inv y = y \in I$ by the definition of an ideal, so $I=F$, so $F$ is a simple ring.
\item The given "hint" is the entire proof.
\item The functions whose range is contained in $\{0, 1\}$ are the idempotents in a ring of functions. The only continuous such functions are the constant 0 and 1 functions.
\item $\twomat{1}{1}{0}{0}$.
\item $re + r'(1-e) = 0$ for some $r, r' \in R \ra re + r' - r'e = 0 \ra e(re + r' - r'e) = 0 \ra re^2 + r'e - r'e^2 = 0 \ra re + r'e - r'e = 0 \ra re = 0 \ra r = 0 \ra r'=0$, so by 6.2.30d, $R \cong Re \oplus R(1-e)$. 
\item 15 is a nontrivial idempotent, and $\ints_{35} \cong 15 \ints_{35} \oplus 21 \ints_{35} \cong \ints_7 \oplus \ints_5$.
\item The colonel of a homomorphism $f$ is an ideal so in a simple ring $R$ it must be either all of $R$ in which case $f=0$, or $\{0\}$ in which case $f$ is injective.
\item Let $A_\alpha$ be a family of ideals in $R$. Then $x \in \cap A_\alpha \ra x \in A_\alpha$ for each $A_\alpha \ra rxr' \in A_\alpha \forall r, r' \in R \forall \alpha$ since each $A_\alpha$ is an ideal, so $rxr' \in \cap A_\alpha \ra \cap A_\alpha$ is an ideal. The other statement is trivial.
\item For $r, r' \in R, a + b \in I + J, r(a + b)r' = rar' + rbr'$. Since $I$ is an ideal, $rar'\in I$, and similarly $rbr' \in J$, so $rar' + rbr' \in I + J$, so $I + J$ is an ideal.
\item For $r, r' \in R, a_i \in I, b_i \in J, r(\sum_i a_ib_i)r' = \sum_ira_ib_ir' = \sum_i(ra_i)(b_ir') \in IJ$ since $ra_i \in I$ and $b_ir' \in J$ since $I$ and $J$ are ideals, so $IJ$ is an ideal. Also, since $a_i \in I, b_i \in R$, $a_ib_i \in I$, and since $b_i \in J, a_i \in R, a_ib_i \in J$, so each $a_ib_i \in I \cap J$, so $IJ \subset I \cap J$.
\item Any ideal that contains $a$ must at least contain all powers of $a$ and all products $ra, ar, rar'$ for $r, r' \in R$, by the definition of ideal, so if this set is an ideal then it is the smallest ideal containing $a$. In fact, $r(a^n + r'a + ar'' + r'''ar'''')r''''' = ra^nr''''' + rr'ar''''' + rar''r''''' + rr'''ar''''r''''' \in \ints a + Ra + aR + RaR$, so $\ints a + Ra + aR + RaR$ is the ideal generated by $a$. If $R$ is commutative, then the last three items in this sum are the same, so $\ints a + Ra + aR + RaR = \ints a + Ra$.
\item For $r, r', r'', r''' \in R, m \in M, r(m + r'ar'')r''' = rmr''' + rr'ar''r''' \in M + RaR$ since $M$ is an ideal, so $M + RaR$ is an ideal. Since any ideal containing $A$ must contain $RaR$, $M + RaR$ is the smallest ideal that contains $M$ and $a$. If $R$ does not have an identity, then the ideal generated by $M$ and $a$ is $M + \ints a Ra + aR + RaR$, since if $1\not\in R$ then any of $RaR$, $Ra$, $aR$ and $\ints a$ might contain elements not present in any other of them, whereas if $R$ has an identity then all of them are contained in $RaR$. All four of $RaR$, $Ra$, $aR$ and $\ints a$ must be present in order for a set containing $a$ to be an ideal.
\item \be
\item $(n, r)((m, s)(l, p)) = (n, r) (ml, mp + sl + sp) = (nml, nmp + nsl + nsp + rml + rmp + rsl + rsp) = (nm, ns + rm + rs)(l,p) = ((n, r)(m, s))(l, p)$, so multiplication is associative, and $(n, r)((m, s) + (l, p)) = (n, r)(m + l, s + p) = (n(m + l), n(s + p) + r(s+p) + r(m+l)) = (nm + nl, ns + np + rs + rp + rm + rl) = (nm, ns + mr + rs) + (nl, np + rl + rp) = (n, r)(m, s) + (n, r)(l, p)$, so multiplication is distributive. The direct product of abelian groups is an abelian group, so $\tilde{R}$ is a ring. $(n, r)(1, 0) = (n, 0 + r+ 0)$, so (1, 0) is the identity in $\tilde{R}$.
\item $f: r \to (0, r)$ is injective, since its colonel is 0, and its image is clearly $0 \times R$. $f(rl) = (0, rl) = (0, r)(0, l) = f(r)f(l)$, so $f$ is a homomorphism. $0 \times R$ is the colonel of the homomorphism $(n, l) \to n$, so it is an ideal.
\item Let $\tf(n, r) = n1 + f(r)$, then $\tf(0, r) = f(r)$ and $\tf(1, 0) = 1$. $\tf$ is a homomorphism since $\tf((n, r)(m, s)) = \tf(nm, ns + mr + rs) = 1nm + f(ns + mr + rs) = 1nm + nf(s) + mf(r) + f(r)f(s) = (n1 + f(r))(m1 + f(s)) = \tf(n, r)\tf(m, s)$, and $\tf((n, r) + (m, s)) = \tf(m + n, r + s) = (m + n)1 + f(r+s) = m1 + n1 + f(r) + f(s) = \tf(n, r) + \tf(m, s)$. Uniqueness follows since our $\tf$ can be derived by starting with $\tf(0, r) = f(r)$ and $\tf(1, 0) = 1$ and applying the homomorphism properties.
\ee
\ee 
\section{Quotient Rings}
\be
\item $x^2 = x^2 - 1 + 1, $ so $(a + bx)(c + dx) = ac + (ad + bc)x + bdx^2 = ac + (ad + bc)x + bd$.
\item $x^3 = x^3 - 1 + 1$, and $x^4 = x(x^3 - 1) + x$, so $(a + bx + cx^2)(a' + b'x + c'x^2) = aa' + ab'x + ac'x^2 + ba'x + bb'x^2 + bc'x^3 + ca'x^2 + cb'x^3 + cc'x^4 = (aa' + bc' + cb') + (ab' + ba' + cc')x + (ac' + bb' + ca')x^2$.
\item Let $B \subset \bar{R}$ be a subring, then $\psi\inv(B)$ is a subgroup of $R$ and if $x, y \in \psi\inv(B)$ then $x = \psi\inv(a), y = \psi\inv(b)$ for some $a, b \in \bar{R}$ so $xy = \psi\inv(a)\psi\inv(b) = \psi\inv(ab) \in \psi\inv(B)$ since $B$ is a ring, so $\psi\inv(B)$ is a ring.

And if $I$ is an ideal in $\bar{R}$ and $x \in \psi\inv(I)$ then for any $r \in R$, $r = \psi\inv(p)$ for some $p \in \bar{R}$ since $\psi$ is surjective so $rx = \psi\inv(p)\psi\inv(x) = \psi\inv(px) \in \psi\inv(I)$, so $\psi\inv(I)$ is an ideal.
\item Let $f(x) = x + \bar{I}$, then, $f \circ \rho$ is a surjective ring homomorphism with colonel $I$ since $\rho(I) = \bar{I}$, so by the homomorphism theorem $R / I \cong \bar{R} / \bar{I}$ using $f \circ \rho$ as the subjective map that the theorem refers to, and furthermore $\bar{R} \cong R / J$ and $\bar{I} \cong I  /J$ by the homomorphism theorem using $\rho$ as the map.
\item Let $x \in R$. Then $x \in \rho\inv(\rho(A)) \lra \exists a \in A:\rho(x) = \rho(a) \lra x \in a + I \lra x \in A + I$, so $\rho\inv(\rho(A)) = A + I$. $A+I$ is a group since it's a sum of groups, and $(a + i)(a' + i') = aa' + (ai' + a'i + ii') \in A + I$ since $I$ is an ideal, so $A + I$ is a subring. $A + I$ contains $I$ since $0 \in A$ since $A$ is a ring, and so the homomorphism theorem applied to the restriction of $\rho$ to $A+I$ gives $(A+I)/I \cong \rho(A)$, and the homomorphism theorem with the restriction of $\rho$ to $A$ gives $A / (A \cap I) \cong \rho(A)$, since the colonel of $\rho$ is $I$. 
\item Let $N$ be an ideal in $R/M$, then $\rho\inv(N)$ where $\rho$ is the quotient projection is an ideal in $R$ containing $M$ by Proposition 6.3.7. If $N$ is a nonzero proper ideal then $\rho\inv(N)$ is a proper ideal in $R$, since $x \not \in N \lra \rho\inv(x) \not \in \rho\inv(N)$, so $M$ is not maximal since this proper ideal contains it. Since $\rho\inv$ is a bijection between ideals of $R/M$ and ideals of $R$ containing $M$, if there is no such proper ideal $N\subset R/M$ then there is no proper ideal of $R$ containing $M$, so $M$ is maximal.
\item \be
\item Let $n=mk$ be a factorization of $n$, then for all $nz \in n\ints$, $nz = mkz$, so $n\ints \subset m\ints$ and $n\ints \subset k\ints$. Iff $n$ is prime, the only such $m$ and $k$ are 1 and $n$, so $n\ints$ is a maximal ideal.
\item Let $f = gh$ be a factorization of $f$, then for all $fq \in fK[x]$, $fq = ghq$, so $fK[x] \subset gK[x]$ and $fK[x] \subset hK[x]$. Iff $f$ is irreducible, the only such $g$ and $h$ are units and polynomials of the same degree as $f$, so $fK[x]$ is a maximal ideal.
\item This follows from Proposition 6.3.13, which states that a ring modulo a maximal ideal is a field, and $n\ints$ and $fK[x]$ are maximal ideals iff $n$ is prime and $f$ is irreducible.
\ee
\item Consider a product $(\sum_j a_jx^j)(\sum_k i_kx^k)$, where $a_j \in R$ and $i_j \in J$. The $l$th coefficient of this product is $\sum_{j+k = l}a_ki_j$; each element in this sum is a product of an element of $I$ with an element of $R$, so each element in this sum is in $I$, so the sum is in $I$, so the product in question is in $J[x]$, so $J[x]$ is an ideal. 

The map $f: \sum_i r_ix^i \to \sum_i (r_i + J)x^i$ is a homomorphism from $R[x]$ to $(R/J)[x]$ by corollary 6.2.9 since the quotient map is a homomorphism. The colonel of this map is $J[x]$, so by the homomorphism theorem for rings, $R[x]/J[x] \cong (R/J)[x]$.
\item Let $P\in \mat_n(R), I \in \mat_n(J)$. Then $(PI)_{ij} = \sum_k P_{ik}I_{kj}$, and each element in that sum is in $J$ since $J$ is an ideal, so the whole sum is in $J$, so the product $PI \in \mat_n(J)$, so $\mat_n(J)$ is an ideal in $\mat_n(R)$.

Define $f: \mat_n(R) \to \mat_n(R/J)$ by $f(A)_{ij} = A_{ij} + J$, then $f$ is a surjective homomorphism since the quotient map respects addition and multiplication, and its colonel is $\mat_n(J)$, so $\mat_n(R) / \mat_n(J) \cong \mat_n(R/J)$ by the homomorphism theorem for rings.
\item Let $f(\sum_i r_ix^i) = r_0$, then $f$ is a surjective homomorphism from $R[x]$ to $R$ with colonel $xR[x]$, so by the homomorphism theorem for rings we have $R \cong R[x] / xR[x]$.
\item \be
\item This map $f$ is injective since if $f(x) = (0 + P, 0 + Q)$, then $x \in P \cap Q = \{0\}$. And it's homomorphic because the quotient map is. For surjectivity, since $P + Q = R$, we have for any $x \in R$ a $p \in P, q \in Q: p + q = x$. So for any $(a + P, b + Q) \in R/P \oplus R/Q$, there is a $p \in P, q \in Q: p + q = a$ and there is $p', \in P, q' \in Q: p' + q' = b$. So let $y = q + p'$. Then $y + P = a + P$, and $y + Q = b + Q$, so $f(y) = (a + P, b + Q)$, so $f$ is surjective. 
\item Let $f(r) = (r + P, r + Q)$ then the same proof as in part a gives that $f$ is an isomorphism from $R$ to $R/P \oplus R/Q$ with colonel $P \cap Q$, and the result follows from the ring homomorphism theorem.
\ee 
\item Let $a_1, a_2, \ldots a_n$ be a sequence of relatively prime polynomials, and let $a = \prod_ia_i$, then $aK[x] \cong a_1K[x] \oplus a_2K[x] \oplus \ldots \oplus a_nK[x]$ under the isomorphism $g: f \to (f + a_1K[x], f + a_2K[x], \ldots f + a_nK[x])$.
\begin{proof}
If $g(q) = 0$, then each $a_i$ divides $q$, so $q$ is a multiple of $a$, so it is zero modulo $a$, so $g$ is injective. $g$ is a homomorphism because the quotient map is. Let $(x_1 + a_1K[x], x_2 + a_2K[x], \ldots x_n + a_nK[x]) \in a_1K[x] \oplus a_2K[x] \oplus \ldots \oplus a_nK[x]$, then we want to find polynomials $q_i$ such that $q_i + a_iK[x] = 1 + a_iK[x]$ and $q_i + a_jK[x] = 0 + a_jK[x] \forall j\not= i$, since if we had such $q_i$ then $\sum_i x_iq_i$ would be the preimage of $(x_1 + a_1K[x], x_2 + a_2K[x], \ldots x_n + a_nK[x])$. So let $n_i = a / a_i$, then $n_i + a_jK[x] = 0 \forall j \not= i$ and $n_i$ is relatively prime to $a_i$, so there exist polynomials $f$ and $g$ such that $fa_i + gn_i = 1$, so let $q_i = gn_i$, then $q_i + a_iK[x] = 1 + a_iK[x]$ as we desire, so we have found the $q_i$ polynomials we seek and the result follows.
\end{proof}
\ee 
\section{Integral Domains}
\be
\item \be
\item If $r, x \in R, x^k = 0$, then $(rx)^k = r^k x^k = r^k0 = 0$, so the set of nilpotents is an ideal.
\item $(r+N)^k = 0 \ra r^k + N = 0 +N\ra r^k \in N \ra r^k$ is nilpotent $\ra r$ is nilpotent $\ra r \in N \ra r + N = 0 + N$. 
\item If $x^k = 0$, then $0 = \phi(x^k) = \phi(x)^k$, so $\phi(x)$ must be zero or else $\phi(x)^k$ would be a product of nonzero elements whose product is zero, which cannot exist in $S$ since it is an integral domain.
\ee
\item Suppose $x^k = 0$. Then computations like $(1-x)(1 + x) = 1 -x^2$; $(1 - x^2)(1 + x^2) = 1 - x^4$, and so forth give a sequence of ring elements $1-x^{2^n}$ divisible by 1-$x$, and you need only repeat until $2^n > k$ to find $1-x^{2^n} = 1-0 = 1$. 
\item $e^2 = e \ra e(e-1) = 0$, so in an integral domain $e$ must be 1 or 0, otherwise this product would be a product of nonzero elements that is equal to 0.
\item Let $f, g \in R[x]\setminus \{0\}$, and let $m, n$ be the degrees of $f$ and $g$. Then the coefficient of $x^{m+n}$ in $fg$ is the product of the coefficients of $x^m$ in $f$ and $x^n$ in $g$; since $R$ is an integral domain, this product is nonzero since neither $m$ nor $n$ is zero by the definition of degree, so $R[x]$ is an integral domain.
\item Exercise 4 showed that $R[x]$ is an integral domain; another previous exercise showed that $R[x, y] \cong R[x][y]$. So suppose $R[x_1, x_2\ldots x_n]$ is an integral domain, then $R[x_1, x_2, \ldots x_{n+1}] \cong R[x_1, x_2\ldots x_n][x_{n+1}]$, so it is a polynomial ring with coefficients in an integral domain, so it is an integral domain.
\item $R[[x]]$ is an integral domain by the same logic we used in exercise 4, except instead of letting $m$ and $n$ be the degrees of $f$ and $g$ we let them be the smallest powers of $x$ in $f$ and $g$. The units in $R[[x]]$ are just the units in $R$, added to an infinite sequence of zeros.
\item \be
\item $R'$ is a commutative ring with identity by hypothesis, and if $x, y \in R'$ have the property that $xy = 0$, then they have that property as elements of $R$, so $x=0$ or $y=0$.
\item The Gaussian integers are a subring of the integral domain of complex numbers that contains 1, so by part a they are an integral domain.
\item The symmetric polynomials are a subring of all the polynomials that contains 1, so by part a they are an integral domain. 
\ee
\item $\ints$ is an integral domain, but $\ints_6$ is not.
\item $a/b \sim a/b$ since $ab = ab$, $a/b \sim c/d \ra ad = bc \ra bc = ad \ra c/d \sim a/b$, and $a/b \sim a'/b', a'/b' \sim a''/b'' \ra ab' = a'b, a'b'' = a''b' \ra ab'a'b'' = a''b'a'b \ra ab'' = a''b \ra a/b \sim a''/b''$.
\item $a/b \sim a'/b', c/d \sim c'/d' \ra a'b = ab', c'd = cd' \ra (dd')ab' = (dd')a'b, (bb')cd' = (bb')c'd \ra (dd')ab' + (bb')cd' = (dd')a'b + (bb')c'd \ra b'd'(ad + bc) = bd(a'd' + c'b') \ra (ad + bc) / bd \sim (a'd' + c'b') / b'd'$, and $ac/bd \cong a'c'b'd' \lra acb'd' = a'c'bd \lra (ab')(cd') = (a'b)(c'd)$, which follows from our hypotheses, so addition and multiplication on $Q(R)$ are well-defined.
\item \be
\item $a/b(c/d * e/f) = a/b(ce / df) = ace / bdf = (ac/bd) * e/f = (a/b * c/d) * e/f$, so multiplication is associative. $a/b(c/d + e/f) = a/b((cf + de) / df) = a(cf + de) / bdf = acf / bdf + ade / bdf = ac / bd + ae / bf = a/b(c/d) +a/b(e/f)$, so multiplication is distributive over addition. $(1/1)(a/b) = a/b$, and $(0/1)(a/b) = 0/b = 0/1$, so $1/1$ is the identity and $0/1$ is 0, so $Q(R)$ is a ring with multiplicative identity. 
\item $a/b + c/d = (ad + bc) / bd = c/d \lra a = 0.$
\item $(a/b)(b/a) = ab/ba =1/1$.
\ee
\item \be
\item Let $f(a) = a/1$, then $f(ab) = ab/1 = a/1 * b/1 = f(a)f(b)$ and $f(a + b) = (a + b) / 1 = a/1 + b/1 = f(a) + f(b)$ and $f(1) = 1/1$, so $f$ is a unital ring homomorphism.
\item Let $g(a/b) = b\inv a$, then the colonel of $g$ is $[0/1]$ so $g$ is injective and $g(a/b + c/d) = g((ad + bc) / bd) = d\inv b\inv(ad + bc) = b\inv a + d\inv c = g(a/b) + g(c/d)$ and $g(a/b * c/d) = g(ac/bd) = d\inv b\inv ac = g(a/b)g(c/d)$ and $g(a/1) = a$ and $g(1/1) = 1$ so $g$ is the injective unital homomorphism we seek. ($b\inv \in F$ since $F$ is a field.)
\ee
\item Let $f((a + bi)/(c+di)) = (ac + bd + (bc-ad)i) / (c^2 + d^2)$, then since the homomorphism is accomplished only by multiplying $(a/bi) / (c+di)$ by $1 = (c-di)/(c-di)$, it is readily apparent that this is an isomorphism.
\item $\exists a, b \in R: (a+J)(b+J) = 0 + J, a+J \not= 0+J, b+J \not= 0 + J \lra a \not \in J, b \not \in J$, but $ab \in J$, so $J$ is prime if and only if there exist no such $a$ and $b$.
\item $d\ints$ is the set of multiples of $d$, so $d$ divides every element of $d\ints$, and so if $d$ is prime then for any product $ab \in d\ints, d$ divides $a$ or $d$ divides $b$, so $a \in d\ints$ or $b\in d\ints$. On the other hand, if $d$ is not prime, then $d=ab$ for some $a, b$ that $d$ does not divide, and since $ab = d1 \in d\ints, d\ints$ is not prime.
\item Let $J$ be a maximal ideal in a ring $R$, then by coronary 6.3.14 $R/J$ is a field, so it is an integral domain, so $J$ is prime by Exercise 6.4.14.
\ee 
\section{Euclidean Domains, Principal Ideal Domains, and Unique Factorization}
\be
\item \be
\item Do repeated division with remainder, each time obtaining a remainder of smaller degree. $g = q_1f + f_1, f = q_2 f1 + f_2, \ldots f_{r-1} = q_{r+1}f_r$. Then $f_r \in gR + fR$, and $f_r$ divides $g$ and $f$.
\item Any common divisor of $f$ and $g$ divides all elements of $fR + gR$, so if $1 \in fR + gR$ then any common divisor of $f$ and $g$ must divide 1 which means it cannot have degree greater than 0 so 1 is a GCD. If $1 \not \in fR + gR$, then it can't be the GCD since part a already showed that the GCD is in $fR + gR$.
\ee
\item \be
\item Let $I$ be an ideal in $R$, and let $d$ be an element of least nonnegative degree in $I$. Then $dR \subset I$ since $I$ is an ideal and for any $p \in I$ we have $p = qd + r$ where the degree of $r$ is less than the degree of $d$ by the Euclidean properties and $r \in I$ since $p \in I$ and $qd \in I$ so since $d$ has the least nonnegative degree that means that $r$ must be zero so $d$ divides $p$ so $I \subset dR \ra I = dR$.
\item Suppose $p = ab$ and $p$ does not divide $a$, then $p$ and $a$ are relatively prime so 1 $\in pR + aR \ra 1 = pr + as$ for some $r, s \in R \ra b = bpd + abs$ and since $p$ divides $p$ and $p$ divides $ab$, $p$ divides the right hand side of this equation so it divides $b$.
\ee
\item We already showed that $R$ is a principal ideal domain, and that implies that it's a unique factorization domain by 6.5.19.
\item Let $N(z) = |z|^2$ for $z \in \ints[\sqrt{-2}]$, then $N$ respects multiplication since the complex norm has that property and given $z, w \in \ints[\sqrt{\-2}]$, let $q$ be the element of $ \ints[\sqrt{\-2}]$ closest to $z/w$ then $|R(q-z/w)| \le 1/2$ and $I(q-z/w)| \le \sqrt{2}/2$, so $z = qw + r$ where $r \in  \ints[\sqrt{\-2}]$ and $N(r) = |z/w-q|^2N(w) \le ((1/2)^2 + (\sqrt{2}/2)^2)N(w) = 3/4 N(w)$ so $ \ints[\sqrt{\-2}]$ is Euclidean.
\item This set is isomorphic to the Gaussian integers under the rotation of the complex plane by $2\pi/3$ radians, so it is also a Euclidean domain.
\item $|ab| = |a||b|$, so the norm of a divisor of $x \in \ints[i]$ divides the norm of $x$, so the norm of a common divisor of $14 + 2i$ and $21 + 26i$ must divide 200 and 1117, which is prime so a GCD is 1.
\item Using the same process, the norm of a common divisor of $33 + 19i$ and $18-16i$ must divide 1450 and 580, so it can have norm up to 290, and $17+i$ has just this norm so it is a common divisor.
\item $a | b \ra b = ac; b | a \ra a = bd \ra b = bdc \ra dc = 1 \ra c$ and $d$ are units. For the reverse implication, $a = ub \lra u\inv a = b$.
\item $a = a1$ so it's reflexive. $a = ub \ra b = u\inv a$ so it's symmetric. $a = ub, b = u'd \ra a = uu'd$, and the product of units is a unit so it's transitive.
\item There are no nonzero nonunits of degree 1, 2 or 3, so the base case is the nonzero, nonunit elements of degree 4, $\pm2$, which are irreducible so they have a factorization by irreducibles. So suppose all nonzero nonunits of degree $\le n$ have a factorization by irreducibles, then for an element of degree $n+1$, $n=ab$ is a proper factorization $\lra |a| < |n|$ and $|b| < |n|$, since $|n| = |a||b|$ and $|x| = 1 \lra x $ is a unit. So for any proper factorization of $n$, the proper factors have a factorization by irreducibles, so either $n$ is irreducible or it has a factorization by irreducibles; in either case, it has a factorization by irreducibles. 
\item Consider $I = \{2x + iy\sqrt{5}\}$, the set of elements of $\ints[\sqrt{-5}]$ whose real part is divisible by 2. Since 2 is irreducible, the only possible $x$ such that $I = x\ints[\sqrt{-5}]$ are $1, -1, 2, $ and $-2$, none of which work.
\item $x\rats[x]$ is an ideal in $\ints + x\rats[x]$, since given $xp(x) \in x\rats[x], n + q(x) \in \ints + x\rats[x], (xp(x))(n + q(x)) = nxp(x) + xp(x)q(x) \in x\rats[x]$, but it is not principal since if $p \in \ints + x\rats[x], p(\ints + x\rats[x])$ is the set of polynomials of degree $d(p)$ or greater whose term of degree $p$ has a coefficient which is an integer times the coefficient of that integer in $p$, and the integer part of $p$ must be 0 or else products of $p$ and elements of $\ints + x\rats[x]$ will have nonzero integer part so $p$ must be a monomial degree 1 to be even remotely viable, but there is no rational $q$ such that $q\ints = \rats$, so there is no $p=qx$ such that $qx(\ints + x\rats[x]) = x\rats[x]$.
\item $a | b \lra b=ax $ for some $x \in R \lra$ for any $r \in R, br = axe \in aR \lra bR \subset aR$. And then if $a$ is a proper factor of $b$ then $b = ax$ where neither $a$ nor $x$ is a unit, so $bR = axR$ and $1 \not \in xR$ since $x$ is not a unit so $a \not \in axR \ra a \not \in bR$, but $a \in aR$ so $bR \subsetneq aR$. And if $a$ is not a proper factor of $b$ then either $a$ does not divide $b$, in which case $b \not \in aR$ so $bR \not \subsetneq aR$, or $b = ax$ where $x$ is a unit so $bR = axR = aR$. You have to suppose $a$ isn't a unit, or the claim doesn't hold.
\item $x |a, x|b \ra x |d \ra x |ud$ for all units $u$, and $ud$ divides $a$ and $b$ for all $u$ since $a=dx \lra ua = udx \lra a = (ud)u\inv x$. If $a$ and $b$ are associates, then $a$ divides $b$ and $b$ divides $a$, so they are common divisors of each other, and any $d$ that divides $a$ and $b$ must divide $a$ and $b$, so $a$ and $b$ are GCDs.
\item Consider the ideal $aR + bR$, then since $R$ is a PID it must equal $dR$ for some $d$. This $d$ is a greatest common divisor of $a$ and $b$ since $aR \subset dR \ra a \in dR, bR \subset dR \ra b \in dR$, and $d \in dR$ so $d = ar + bs$ for some $r, s \in R$ so any $x \in R$ that divides $a$ and $b$ also divides $ar + bs = d$. Since the GCD in a PID is unique up to multiplication by units, any GCD of $a$ and $b$ must be in $dR = aR + bR$.

$1 \in aR + bR \lra aR + bR = 1R \lra d=1$. 
\item Since $a$ is irreducible, the only nonunit that divides $a$ is $a$, so if $a$ does not divide $b$ the no nonunit divides both $a$ and $b$, so $a$ and $b$ are relatively prime.
\item Let $x \in abR$. Since $a \in R$, $x \in bR$ and vice versa, so $abR \subset aR \cap bR$. And if $y \in aR, y \in bR$, then $y = ax_1 = bx_2$ for some $x_1, x_2 \in R$, so since $a$ does not divide $b$ it must divide $x_2$, so $x_2 = ax_2' \ra y = bax_2' \in abR$, so $aR \cap bR \subset abR$, so $abR = aR \cap bR$. 

$aR + bR = 1R = R$ by exercise 16.

Let $f(x) = (x + aR, x+bR)$ where $x \in R$, then $f$ is a homomorphism from $R$ to $R/aR \oplus R/bR$ with colonel $abR$, so $R/abR \cong R/aR \oplus R/bR$. 
\item \be
\item The factors of $x$ are $x$ times elements of $K$, and the factors of $y$ are $y$ times elements of $K$, and since $y$ does not divide $x$ nor does $x$ divide $y$, their only common factors are elements of $K$ which are units.
\item The constant term of $xp(x)$ is 0, as is the constant term of $yp(y)$, so the constant term of their sum is 0 $\not= 1$. 
\item $xK[x, y] + yK[x, y]$ is an ideal, so if $K[x, y]$ were a PID then it would be principal, but by part b it is not, so $K[x, y]$ is not a PID.
\ee
\item If $p$ is not irreducible then $p = ab$ where $a$ and $b$ are not units, so if $p$ is prime then it divides $a$ or $b$ so supposing without loss of generality that $p$ divides $a$ we have $a = a'p \ra p = ab = pa'b \ra b$ is a unit, so $p$ cannot be prime if it is not irreducible, so all primes are irreducible.
\item If $k=2$, this is the definition of primality. Suppose the statement holds for some $k$, then $b_1b_2\cdots b_{k+1} = (b_1)(b_2\cdots b_{k+1})$ so $p$ divides either $b_1$ or $b_2\cdots b_{k+1}$ and in either case it divides one of the $b_i$.
\item If $f$ is the quotient map $x \to x + aR$, then $f\inv$ maps ideals of $R/aR$ to ideals of $R$ containing $aR$, so if there are infinite ideals in $R/aR$ then there are infinite ideals in $R$ that contain $aR$, but by exercise 13 $aR \subsetneq bR \lra b $ is a proper factor of $a$, so there are only as many ideals in $R$ containing $aR$ as there are proper factors of $a$ plus one for $aR$ itself plus one for $R$, so there are only finitely many ideals since no element of a PID has an infinite number of proper factors.
\item \be
\item $p$ not prime $\lra \exists a, b: p \not | a, p \not | b, p | ab \lra, a + pR \not = 0 + pR, b + pR \not = 0 + pR, ab + pR = 0 + pR \lra R/pR$ not an integral domain.
\item By 6.3.14, $R/M$ is a field iff $M$ is maximal, and by 6.5.18, $pR$ is maximal iff $p$ is irreducible in a PID, so putting these two together we get $R/aR$ is a field if and only if $a$ is irreducible.
\item All fields are integral domains. If $R/I = R/aR$ is an integral domain, then by part a, $a$ is prime, so $a$ is irreducible, so by part b, $R/aR = R/I$ is a field.
\item This is part of Lemma 6.5.18.
\ee
\item \be
\item Let $A = \sum_ia_ix^i$ be a power series, then we need a sequence $B = \sum_jb_jx^j$ such that for each $n \ge 1$, $(AB)_n = 0$, and $(AB)_0 = 1$. We will construct this sequence inductively. Let $b_0 = a_0\inv$, then since $(AB)_0 = a_0b_0$, this is the only possible value for $b_0$. Now suppose the first $n$ terms of our $B$ series are defined, so that $(AB)_0 = 0$ and $(AB)_k = 0$ for $1 \le k \le n$. The coefficient of $x^{n+1}$ in $AB$ is $A_0B_{n+1} + \sum_{i=1}^n A_iB_{n+1-i}$, so setting $B_{n+1} = -A_0\inv(\sum_{i=1}^n A_iB_{n+1-i})$ gives us the desired properties. This defines the inverse of any sequence with a nonzero constant term, so those are units in $K[[x]]$, and it is clear that no other element is a unit since if the constant term of $A$ is 0 then the constant term of $AB = 0 \not=1$ for all $B$. 
\item Let $J$ be an ideal of $K[[x]]$, and let $n$ be the least integer such that $J$ has an element of the form $a_nx^n + \sum_{j>n} a_jx^j$ where $a_n \not = 0$. Then $J \subset x^nK[[x]]$. Furthermore, $a_nx^n + \sum_{k>n}a_kx^k = x^n(a_n + \sum_{k>0}a_{k+n}x^k)$, and $A = a_n + \sum_{k>0}a_{k+n}x^k$ is a unit by part a, so for any $B = x^n\sum_i b_i x^i \in x^nK[[x]], B = x^nAA\inv \sum_i b_ix^i$, so $x^nK[[x]] \subset J$, so $J = x^nK[[x]]$. 
\item $x$ is prime so $xK[[x]]$ is maximal by 6.5.18, and if $I$ is some maximal ideal in $K[[x]]$ then $I = x^nK[[x]]$ for some $n$ but the only $n$ such that $x^nK[[x]]$ is prime is 1, so $xK[[x]]$ is unique. $K[[x]] / xK[[x]] \cong K$ since $K[[x]] / xK[[x]]$ it consists of equivalence classes $k + xK[[x]]$ for all $k \in K$.
\ee
\item Let $I$ be an ideal in $\rats_p$, and let $p^k$ be the smallest value of $p$ such that $I$ contains an element $p^ka/b$ where $p$ does not divide $a$. Then $b/a \in I$, so $p^k \in I$, so $p^ka'/b' \in I$ for all $a'/b' \in \rats_p$, so $p^k\rats_p \subset I$. By assumption,  $I \subset p^k\rats_p$, so $I = p^k\rats_p$, so $I = p^k\rats_p$, so $I$ is principal.
\ee 
\section{Unique Factorization Domains}
\be
\item \be
\item The GCD of several elements $a_i$ in a UFD is constructed in the proof of lemma 6.6.2; it is the product of the intersection of the common nonunit factors of all the $a_i$. It is readily apparent that the intersection of the common nonunit factors of the $ba_i$ is the union of $b$ and the intersection of the common factors of the $a_i$.
\item If $f$ is primitive then 1 is the GCD of its coefficients so $b1 = b$ is the gcd of $b$ times its coefficients by part a.
\ee
\item \be
\item Both rings consist of linear combinations of powers of the $x_i$, and they have the same addition and multiplication rules, so they are the same.
\item 6.6.7 is the base case, and part a is the inductive step.
\ee
\item $p(r/s) = 0 \lra (x-r/s) | p \lra sx-r | p$ so if $q = \sum_iq_ix^i$ is $p/(sx-r)$ then $p_n = sq_n \ra s|p_n$ and $p_0 = rq_0 \ra r|p_0$.
\item We didn't use any special properties of the integers in proving the previous exercise, just the general properties of fields of fractions and Gauss's lemma.
\item \be
\item $\pi_p(h) = 0 \lra \forall i, \pi_p(h_i) = 0 \lra \forall i, p | h_i$.
\item $\pi_p(f) = 0 \lra \forall i, p | f_i \lra \exists$ an irreducible $p$ dividing all coefficients of $f \lra f$ is not primitive.
\item $p$ is prime $\ra pR$ is prime $\ra R/pR$ is an integral domain by 6.4.11.
\item Iff an irreducible element $p$ divides all the coefficients of $fg$, then $fg + pR = 0 + pR$, so either $f+pR = 0 + pR$ or $g+ pR = 0 + pR$ since $R/pR$ is an integral domain by part c, so $p$ divides all the coefficients of $f$ or all those of $g$ by part a, so one of $f$ and $g$ is not primitive. Thus, $fg$ is primitive $\lra f$ and $g$ are primitive. 
\ee
\item 3 is irreducible but not prime since it divides $6 = (1 + \sqrt{-5})(1- \sqrt{-5})$. But $\ints[\sqrt{-5}]$ satisfies the ACC since $a\ints[\sqrt{-5}] \subset b\ints[\sqrt{-5}] \lra b | a$. That's because if $a=bx$ then for any $ay \in a\ints[\sqrt{-5}]$, $ay = bxy \in b\ints[\sqrt{-5}]$. Whereas if $b$ does not divide $a$ then $a \not \in b\ints[\sqrt{-5}]$.
\item If $f(x) = z + xp(x) \in \ints + x\rats[x]$ is irreducible in $\ints + x\rats[x]$, so it is irreducible in $\ints[x] \subset \ints + x\rats[x]$, so it is irreducible in $\rats[x]$ by Gauss's lemma, so it is prime in $\rats[x]$ since $\rats[x]$ is a UFD, so it is prime in $\ints + x\rats[x] \subset \rats[x]$.

The sequence $A_n = x/2^n(\ints + x\rats[x]), n=1, 2, 3\ldots$ is an infinite sequence of increasing principal ideals in $\ints + x\rats[x]$, since $A_{n}$ does not contain the polynomials whose $x$ coefficient is $1/2^{n+1}$ whereas $A_{n+1}$ does contain them, and $A_n \subset A_{n+1}$ since $2A_{n+1} = A_n$ and $2 \in \ints + x\rats[x]$.
\ee 
\section{Noetherian Rings}
\be
\item Let $ax^k \in J$, then for all $r \in R$, $arx^k \in J$ since $J$ is an ideal, so $aR \subset A_k$, so $A_k$ is an ideal. $A_k \subset A_{k+1}$ because for all $bx^k \in J$, $xb^k = bx^{k+1} \in J$, again since $J$ is an ideal.
\item Let $J$ be an ideal in $S$, then by 6.3.7, $I = \psi\inv(J)$ is an ideal in $R$, so $I$ is finitely generated since $R$ is Noetherian, so if $\{a_1, a_2 \ldots a_k\}$ are the generators of $I$ then $\{\psi(a_1), \psi(a_2), \ldots \psi(a_n)\}$ are the generators of $J$ since for $j \in J$ we have $j = \psi(i)$ for some $i \in I$, and $i$ is a product of some of the $a_k$ since they generate $I$, and $\psi$ is a homomorphism so it respects this multiplication. Thus, any ideal in the range of $\psi$ is finitely generated, and since $\psi$ is surjective that means all of the ideals in $S$ are finitely generated, so $S$ is Noetherian.
\item For $f \in \ints[x]$, let $\psi(f) = f(\sqrt{-5})$, then $\psi$ is a homomorphism since it is an evaluation of a polynomial which was shown to be a homomorphism in 6.2.7, and $\psi$ is surjective since $\psi\inv(a + b\sqrt{-5}) = \psi(a + bx)$, so $\ints[\sqrt{-5}]$ is Noetherian by 2.
\item We showed in Exercise 6.6.7 that principal ideals don't satisfy the ascending chain condition in $\ints + x\rats[x]$, so it is not Noetherian.
\item If $R$ is Noetherian, then all increasing sequences of ideals are finite, so in particular all increasing sequences of principal ideals are finite. This plus every irreducible element is prime means that $R$ is a unique factorization domain by Proposition 6.6.16.
\ee 
\section{Irreducibility Criteria}
\be
\item By Proposition 1.8.22, $f$ has no factor of degree 1 if it has no rational root. So if $f=pq$, where $p, q \in \ints[x]$ have degree between 1 and $n-1$ exclusive, then $\pi_p(f) = \pi_p(a)\pi_p(b)$, and by assumption $\pi_p(f) = (x+r)(q)$ for some $r \in \ints_p, q \in \ints_p(x)$. But so then modulo $p$, $ab = (x+r)q$, so since $q$ is irreducible by hypothesis it must divide $a$ or $b$, which it can't because their degree is lower than $q$'s, so there can exist no such factorization $ab$ of $f$.
\item $f(x+1) = \sum_{n=0}^{p-1}(x+1)^n = \sum_{n=0}^{p-1}(\sum_{k=0}^n$$n\choose k$$x^k)$ by the binomial theorem. So the coefficient of $x^k$ in the expanded product is $\sum_{n=0}^{p-1}$$n\choose k$$ = \sum_{n=k}^{p-1}$$n\choose k$$ = $$p \choose {k+1}$, so $p$ divides each coefficient except when $k=p-1$ and the coefficient is 1, and $p^2$ does not divide the zeroth coefficient which is $p \choose 1$ = $p$, so by Eisenstein's criterion $f$ is irreducible.
Lemma: $\sum_{n=k}^{p-1}$$n\choose k$$ = $$p \choose {k+1}$$ \forall p \in \nats \forall k < p$. \begin{proof}
For $p=1$, $\sum_{n=0}^0 $$n \choose 0$$ = 1 = $$1 \choose 1$. And then if the lemma holds all the way to $p-1$, then $\sum_{n=0}^{p-1}$$n\choose k$$ = \sum_{n=0}^{p-2}$$n \choose k$$ + $$p-1 \choose k$$ = $$p-1 \choose k-1$$ + $$p-1 \choose k$$ = $$p \choose k+1$ by 1.9.3c.
\end{proof}
\item Let $f_n(x)$ be the polynomial under consideration. So $f_n(k) = -1$ for $k = 1, 2, 3, \ldots n$, so if $f_n(x) = g(x)h(x)$ for some proper factors $g$ and $h$ (which then must have degree less than deg($f_n) = n$), then $g(k) = \pm 1, h(k) = \mp 1$ for $k = 1, 2, 3, \ldots n$. So $g(k) + h(k) = 0$ at all these points, so $g+h$ has $n$ zeros, and since the degree of $g + h$ is less than $n$, $g + h = 0 \ra g = -h$. So $f_n(x) = -g(x)^2$, but the degree-$n$ term of $f_n$ is 1, and there is no integer polynomial $g$ whose square has degree-$n$ term -1, since the degree-$n/2$ term of $g$ would have to be $\sqrt{-1}$, so $f$ has no proper factorization.
\item The same train of logic as in exercise 3 leads us to the conclusion that $f_n(x) = g(x)^2$ for some polynomial $g(x)$, if $f_n$ has any proper factorization at all. Beyond that, though, you're on your own.
\item Computers can do this crap well enough that I don't need to bother with it.
\ee 
\end{document}
